\chapter{Variations and Generalisations}

\todo{- several variations of the current design and the implications / use-cases
- generalisations if the current design would be extended to a cache-like architecture
}

\todo{peter comment during call on 18 september:\\
- generalisations part; there is a continuum between making observations, outlining future work, actually solving something
}

\todo{- opencapi returns data out of order. now we reorder before L2, but you could make everything out of order.}

\todo{- variations and generalisations chapter: show that multiple streams could also be used for strided access or other patterns i generalized over.\\
- out of order execution of reads/writes to not stall and keep up with the bandwidth\\
- multiple read ports allow for multiple AFU engines to work concurrently. use decompress-filter use case as example. also think about other ways of using 8 read ports, see slides for IBM.\\
- variation is to use multiple independent operations on the same FPGA. then multiple streams are also needed. for example, half of the streams for AFU 0 and the rest for AFU 1.\\
- you can extend the read ports to more than eight as well, lets call them virtual read ports, by adding arbitration logic.\\
}

\todo{- partition the FPGA as multiple different accelerators. need multiple streams to feed them concurrently.\\
- interested in the interface, not the partition of compute of the AFU.\\
- interesting to explore how the different architectures can be serviced with this interface in variations chapter.\\
  - example is different AFUs on same FPGA. requires multiple streams\\
  - variation is also to implement single stream, then interface is like a DMA with cache line granularity, but possibility to read across a cache line boundary.\\
  - pipeline compute on FPGA is also option, but doesn't change anything for interface.\\
}
